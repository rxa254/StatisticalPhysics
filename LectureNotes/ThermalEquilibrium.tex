\section{Equilibrium: April 7}

\subsection{Review of Last Week}
\begin{itemize}
\item \textbf{Fundamental Assumption of Statistical Mechanics:} "An isolated system in uqilibrium is equally likely to be found in any of the microstates available to it."

\item $g =$ "the multiplicity"; the number of accessible micro states for a given set of extensive parameters (e.g. U, N, V)

\item If all such states are equally likely, then \textit{the probability distribution is uniform}: $P_i = 1/g$

\item The mean value or "Expectation value" of an observable $\mathcal{A}$ is defined 
  $\Braket{\mathcal{A}} = \displaystyle \sum_{i} \mathcal{A}_i P(i)$

\item Fluctuations are \textit{\textbf{tiny}}: For N >> 1, quantities like $\Delta U/U$ go as $1/\sqrt{N}$

\item This is a consequence of the Central Limit Theorem. Ensembles of many varying distributions sum up to a Gaussian distribution.

\end{itemize}

\subsubsection{Collection of N magnetic spins}
\begin{equation}
g(N,s) = \frac{N!}{(N/2 + s)! (N/2-s)!}
\end{equation}
Using Stirling's Approximation (cf. \url{http://mathworld.wolfram.com/StirlingsApproximation.html}):
\begin{equation}
log(N!) = N log(N) - N + \frac{1}{2}log(2 \pi N)
\end{equation}
\begin{figure}[h]
\centering
\includegraphics[width=\columnwidth]{Figures/Stirling.pdf}
\caption{Comparison of various approximations to N!}
\end{figure}
after some substitutions, we find that

\begin{equation}
g(N,s) = 2^N \sqrt{\frac{2}{\pi N}} exp\bigg[-\frac{1}{2}\bigg(\frac{s}{\sigma_s}\bigg)^2\bigg]
\end{equation}
where $\sigma_s = \sqrt{N}/2$ is the standard deviation of the Gaussian probability 
distribution of $s$, the spin excess.

\subsection{Thermal Equilibrium for Two Systems}
Let us consider two non-interacting systems. Each system is a box with a collections of
spins. Box 1 has $N_1$ total atoms and an initial spin excess of $s_1$.
The combined multiplicity of these 
two systems (before we allow them to interact) is (using the same logic that we used to consider
our binary model of coin flips) just the product of the individual multiplicities:
\begin{align}
g_{tot} &= g_1 \times g_2 \\
        &= \frac{2}{\pi} \frac{1}{\sqrt{N_1 N_2}} 2^{N_1 + N_2} exp\bigg[-2\bigg(\frac{s_1^2}{N_1} + 
        \frac{s_2^2}{N_2}\bigg)\bigg]
\label{eq:ginit}
\end{align}
Rather than consider some artificial magnetic spin exchange interaction, let us instead
turn on an external magnetic field, such that the energy of each system becomes
$U_i = -2 m B s_i$, where $m$ is the magnetic moment of each atom and $B$ is the magnetic
field. We then bring the boxes into contact, such that it is possible to interchange energy
between the two system. The total spin and the total energy will remain conserved
(e.g. $U = U_1 + U_2$). For this two system example, we do not let the number of atoms in
each box change: $N_1 = const, N_2 = const$.

After the energy exchange begins, the two systems move from their 
initial state (Eq.~\ref{eq:ginit}) into a new state
\begin{equation}
g_{tot} = g1(U_1^\prime) \times g2(U_2^\prime)
\end{equation}
with more accessible microstates.

We would like to find what the new equilibrium state is. In other words, what is the most probable
state after the two systems have been in contact long enough to come into equilibrium?
\begin{figure}[h]
\centering
\includegraphics[width=\columnwidth]{Figures/TwoSystemMultiplicity.pdf}
\caption{The multiplicity functions for two systems before (Blue and Red) and after (Purple) being put 
	into thermal contact.}
\end{figure}

To find this, we would like to find the stationary point for the combined multiplicity function.

\begin{equation}
dg = \bigg(\frac{\partial g_1}{\partial U_1}\bigg)_{N_1} g_2~dU_1 +
     \bigg(\frac{\partial g_2}{\partial U_2}\bigg)_{N_2} g_1~dU_2 = 0  
\label{eq:maxg}
\end{equation}

\begin{equation}
\bigg(\frac{\partial log(g_1)}{\partial U_1}\bigg)_{N_1} = 
\bigg(\frac{\partial log(g_2)}{\partial U_2}\bigg)_{N_2}
\label{eq:maxmicro}
\end{equation}
