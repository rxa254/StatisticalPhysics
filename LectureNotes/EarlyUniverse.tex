\section{The Early Universe: May 26}

\subsection{The Big Bang}

\subsection{The First Three Seconds}

\subsection{The First Three Minutes}

\subsection{Recombination}
Once the expansion of the universe causes the temperature to get low enough, the number of photons available to ionize hydrogen becomes low enough that the electrons and protons in the plasma are able to form neutral hydrogen.
\begin{equation}
p + e^- \longrightarrow H
\end{equation}

\subsection{The Cosmic Background Radiation}
This happens at a redshift of $z \sim 1100$, when the universe was only 400000\,years old.

\subsection{The Epoch of Reionization}
The Epoch of Reionization refers to a time much later in the thermal history of the universe when the neutral gas went from almost completely neutral to very ionized. It is suspected that
universe remained neutral after the time of Recombination unit a $z$ of $\sim 10-20$. At this point the formation of the first stars in the universe would have produced enough UV radiation to liberate the electrons from their $-13.6 eV$ (these UV photon are $\sim$4-5 times more energetic than light at the purple end of the rainbow) potential well in hydrogen. 
\begin{figure}[h]
\centering
\includegraphics[width=\columnwidth]{Figures/21-centimeter-cosmology.jpg}
\caption{History of the universe, highlighting the Epoch of Reionization.
credti: \href{http://discovermagazine.com/2014/april/12-first-light}{Roen Kelly, Discover Magazine, April 2014}}
\label{fig:EoR}
\end{figure}

\subsection{Evidence for Dark Matter and Dark Energy}

\subsection{Further Reading}
\begin{itemize}
\item \href{http://discovermagazine.com/2014/april/12-first-light}{Discover article on Epoch of Reionization}
\item \href{http://en.wikipedia.org/wiki/Physical_cosmology}{Wikipedia: Physical Cosmology}
\end{itemize}

