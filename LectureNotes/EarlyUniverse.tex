\section{The Early Universe: May 26 - June 2}
Now that we have covered several of the techniques for deriving thermodynamic observables from the fundamentals of statistical mechanics, we can apply these tools to describe some of the features of the early universe (post Big Bang).


\subsection{The Big Bang}
The actual Big Bang and the first $\sim 10^{-30} s$ after it, are not well understood. The belief is that the universe started from a very hot and very dense state and evolved over billions of years into the state that we now observe. What happened before the big bang?, what made it bang?, was it the only big bang?, and are there going to be other similarly big bangs?, are some of the many unresolved questions in modern cosmology today.

\subsection{The Universe is Expanding}
In 1929, Edwin Hubble used the 100" Hooker Telescope, at the Mount Wilson Observatory in the hills north of Caltech, in combination with other
measurements~\footnote{Henrietta Leavitt's observation of Cepheid variables were key in making the extra galactic calibrations, as were the measurements by
Slipher and Humason} to discover that there was a linear relationship between the distance to a faraway galaxy and its observed recessional velocity.
\begin{figure}[h]
\centering
\includegraphics[width=0.75\columnwidth]{Figures/Hubble1931.png}
\caption{Data from Humason and Hubble (1931). Data below 2\,Mpc from
	the original Hubble (1929) paper.}
\label{fig:Hubble}
% REPLACE w/ real data replotted
\end{figure}
The distance determination was made through a complicated chain of intermediate calibrations (the "galactic distance ladder") and the velocity by measuring the wavelengths of the observed spectra lines and comparing them to the spectral 'fingerprint' of molecules at rest on the earth.\\

Since the expansion velocity is proportional to the separation, we 
can write the distance as $D(t) = a(t) D(t_0)$, where $D(t_0)$ is 
the initial separation and $a(t)$ is the \textit{universal scale factor}.
The gravitational acceleration at the edge of a spherical region of
the universe is $\ddot{D}(t) = -G M / D(t)^2$, where $G$ is Newton's gravitational constant, $M = \frac{4}{3} \pi \rho(t) D(t)^3$ is the mass enclosed within the shell, and $\rho(t)$ is the mass-energy density. If the kinetic energy of the shell just equals the gravitational potential, then the density will be
\begin{equation}
\rho_{crit} = \frac{3 H^2}{8 \pi G}
\label{eq:CriticalDensity}
\end{equation}
where $H \equiv \dot{a}/a$ is the Hubble parameter ($H_0 \simeq 69~(km/s)/Mpc$). If the enclosed density is less than $\rho_{crit}$ (known as the critical density or altenatively as the closure density), then the universe will expand forever. If its more than $\rho_{crit}$, then then gravitational well is too deep and the universe will collapse into a 'Big Crunch'.

From Chapter 4, we know (cf.~\cref{eq:PlanckSpectralDensity}) that the energy density, per unit frequency interval, of blackbody radiation is:
\begin{equation}
u_{\omega} = \frac{\hbar}{\pi^2 c^3} \frac{\omega^3}{e^{\hbar \omega/\tau} - 1}
\end{equation}
and $\Braket{n}$, the mean number of photons per mode, is given by \cref{eq:PlanckOccupy}. Integrating over all frequencies we find that the total energy density is
\begin{equation}
u_{rad} = U/V = 4 \sigma_B \frac{\tau^4}{c}
\end{equation}
where $\sigma_B$ is the Stefan-Boltzmann constant. 







\subsection{Inflation}
Invented to solve several problems: flatness, isotropy, and ....



\subsection{The First Three Minutes}
When the universe is still very hot and less than a few milliseconds old,
the density is dominated by relativistic particles (e.g. photons, neutrinos).
Ignoring for the moment all but the photons, we can estimate how the temperature would evolve as a function of time.
Using \cref{eq:CriticalDensity} and our argument from \cref{sec:CBR} below for the scaling of $\tau$, we find (after some algebra) that:
\begin{align}
H(t) &= \frac{\dot{a}}{a} = \frac{32 \pi G \sigma_B \tau^4}{3 c^3} \\
%a~da &= \sqrt{\frac{32 \pi G \sigma_B}{3 c^3}}~dt \\
\tau &= \bigg(\frac{3 c^3}{32 \pi G \sigma_B}\bigg)^{1/4} \frac{1}{\sqrt{t}}
\end{align}
So in this early, radiation dominated era, $\tau \propto 1/\sqrt{t}$. In SI units, we can write this as (after plugging in numbers):
\begin{equation}
T \simeq 10^{10}~K~\sqrt{\frac{1 s}{t}}
\end{equation}
This (correct) value is a factor of $\sim 2$ lower than what we would get from using the formula above. The difference is that we have to account for other relativistic particles (such as neutrinos) to get the right answer.

At these early and hot times, the relative abundance of neutrons and protons is given by the Boltzmann factor:
\begin{equation}
\frac{n_n}{n_p} \approx exp\bigg(-\frac{\epsilon_{np}}{\tau}\bigg)
\label{eq:NeutronAbundance}
\end{equation}
where $\epsilon_{np} = (m_n - (m_p + m_e))c^2 \sim\,0.7\,MeV$ is the 
difference in the rest energies of the two states (neutron and proton + 
electron). This ratio will be close to unity as long as $T \gg 10^{10}\,K$.
\begin{figure}[h]
\centering
\includegraphics[width=0.75\columnwidth]{Figures/NeutronAbundance.pdf}
\caption{Relative Neutron/Proton abundance as a function of temperature 
(or energy = $k_B T$) in the early universe.}
\label{fig:Neutrons}
\end{figure}

\subsubsection{Nucleosynthesis}
As the universe continued to cool eventually it reached a state where nucleons could form some 
of the light elements 
(e.g. $\prescript{1}{}H$, $\prescript{2}{}H$, $\prescript{3}{}He$, $\prescript{4}{}He$) 
but when and at what rate? 
By $\sim\,1\,s$ after the Big Bang, the temperature had become too low to maintain the 
proton-neutron balance and the fraction slowly drops as shown in~\cref{fig:Neutrons}.

In order for stable elements like helium and lithium to form, the free protons and neutrons 
must first form deuterons (1 proton + 1 neutron) which could then form $\prescript{4}{}He$ 
through the $d + d \rightarrow \prescript{4}{}He + \gamma$ reaction (which has a few 
intermediate steps involving the production of $\prescript{3}{}H and \prescript{3}{}He$). 

However, the binding energy of the deuteron is $\epsilon_{d} = (m_d - (m_p + m_n))c^2 \sim\,2.2\,MeV$
and so this process is suppressed as long as there are enough 2.2\,MeV photons around to 
disrupt the deuterons.
\begin{align}
\prescript{2}{}H + \gamma \longrightarrow p + n
\end{align}

How to calculate the relative abundanced quantitatively? Well, we know that for a system in thermal and chemical equilibrium, the equilibrium condition (cf.~\cref{eq:ChemEquil}) is that:
\begin{equation}
\mu_p + \mu_n = \mu_d + \mu_{\gamma}
\end{equation}
where $\mu_p, \mu_n, \mu_d, \mu_{\gamma}$ are the chemical potentials of the protons, neutrons, deuterons, and photons, respectively. We know that $\mu_{\gamma} = 0$ for blackbody radiation and so it just remains for us to write down expressions for each of the other three terms and solve for the deuteron number density. Since the number density is still below, $n_Q$, the quantum concentration value (cf.~\cref{eq:QuantConc})
\begin{equation}
\mu_i = m_i c^2 + \tau \log{\frac{n_i \lambda_i^3}{g_i}}
\end{equation}
where $n_i$ is the number density, $\lambda_i^3 = 1/n_Q$ is the thermal de Broglie wavelength, and $g_i$ is the degeneracy (protons and neutrons have two spin states, and deuterons have three spin states: -1, 0, +1). The first term in the equation, $m_i c^2$, is the rest energy of the particle. We are including this relativistic energy in the chemical potential to account for the fact that the deuteron has a non-zero binding energy. Using this formula in our detailed chemical balance equation, we find:
\begin{align}
\log{\frac{n_p \lambda_p^3}{2}} + \log{\frac{n_n \lambda_n^3}{2}} - \log{\frac{n_d \lambda_d^3}{3}} = -\epsilon_B/\tau \\
\bigg(\frac{n_p \lambda_p^3}{2}\bigg)\bigg(\frac{n_n \lambda_n^3}{2}\bigg)\bigg(\frac{n_d \lambda_d^3}{3}\bigg)^{-1} = e^{-\epsilon_B/\tau} \\
\frac{n_d}{n_n} = \frac{3}{4} n_p \bigg(\frac{\lambda_p \lambda_n}{\lambda_d}\bigg)^3 e^{\epsilon_B/\tau}
\end{align}
which can be further simplified by noting that $m_p \simeq m_n \simeq m_d/2$. Since this occurs after the neutron-proton ratio 'freeeze out', this equation predicts that virtually all of the neutrons will be used up producing deuterons (which
in turn will end up forming helium).\\

If instead of three species, we had been considering the two species case (balance between protons and neutrons), most of the prefactors would cancel and we would recover the simple expression of \cref{eq:NeutronAbundance}.


\begin{figure}[h]
\centering
\includegraphics[width=0.45\columnwidth]{Figures/relabund.jpg}
\caption{Predicted relative abundances of the light elements as a
a function of $\Omega_B$, the baryon density relative to the 
critical density.}
\label{fig:Neutrons}
\end{figure}


\subsection{Recombination}
While the universe is still hot, the electron and proton gas/plasma is in thermal equilibrium with the photons. They interact because the photons are able to scatter strongly off the electrons\footnote{and the protons to a lesser extent, since the Thompson scattering cross section is proportional to 1/m$^2$. The Thompson scattering approximation is valid as long as the photon 
energy ($\sim 1\,eV$) is much less than the rest mass of the 
electron ($0.5\,MeV$).} since they are charged.

As the universe cools through expansion, the number of photons available to ionize 
hydrogen becomes low enough that the electrons and protons in the plasma are able to 
form neutral, atomic hydrogen ($\prescript{1}{}H$). At what temperature does this happen? Well, the
binding energy of atomic Hydrogen, is one Rydberg 
($1 Ry = \frac{m_e e_c^4}{8 h^2 \epsilon_0^2}$ = 13.6\,eV), so a reasonable guess might be that
$\prescript{1}{}H$ forms when $T < 13.6 eV / k_B$, but in fact it happens at even lower temperatures. 
\begin{align}
p + e^- \longrightarrow \prescript{1}{}H + \gamma
\end{align}
At what temperature does production of neutral hydrogen begin? It must be at the temperature where there are not enough photons (with $h \nu = 13.6\,eV$) to disassociate a significant fraction of the hydrogen atoms. 
From \cref{eq:PlanckOccupy}, we can find the number density of 
blackbody photons as a function of temperature and wavelength.


\begin{figure}[h]
\centering
\includegraphics[width=0.5\columnwidth]{Figures/TemperatureCosmic.pdf}
\caption{Evolution of Temperature as a function of time throughout the history of the universe.
Notice the change in slope between the radiation and matter dominated eras.}
\label{fig:TempVsTime}
\end{figure}



\subsection{The Cosmic Background Radiation}
\label{sec:CBR}
This happens at a redshift of $z \sim 1089$, when the universe was only 380,000\,years old.\\

The isentropic expansion of the universe leaves $\Braket{n}$ unchanged, but the wavelength scales as $\lambda \propto a(t)$. For the mode occupancy to remain constant the denominator of \cref{eq:PlanckOccupy}, must also remain constant and so the blackbody temperature scales as $\tau \propto 1/a(t)$. \\


So now we can compute how many photons, as a function of energy, are available to participate in nucleosynthesis at any given point in the expansion history of the universe. By measuring the temperature of cosmic radiation at present times (2.725\,K; see \cref{fig:FIRAS}) and knowing what the temperature must be at the time of recombination ($\approx$\,3000\,K) we can find that the redshift value at the time of \textit{last scattering} must be:
\begin{align}
z &= T/T_0 - 1 \\
  &= \frac{3000\,K}{2.725\,K} - 1 \\
  &\simeq 1100
\end{align}

\begin{figure}[!h]
\centering
\includegraphics[width=0.75\columnwidth]{Figures/COBE_FIRAS.pdf}
\caption{Cosmic Background Radiation spectrum, as measured with the FIRAS instrument on board the COBE satellite. Data from NASA: \href{http://lambda.gsfc.nasa.gov/product/cobe/firas_prod_table.cfm}{FIRAS Monopole Spectrum}}
\label{fig:FIRAS}
\end{figure}

\begin{figure}[!h]
\centering
\includegraphics[width=0.5\columnwidth]{Figures/cobe53ghz.png}
\caption{Sky Maps from the Cosmic Background Explorer (COBE). The dipole component due to the rotation of the galaxy is $\sim\,0.1\%$ of the monopole (the all sky average). The 10 parts per
million fluctuations at higher spatial scales revealed the quantum density fluctuations from the
inflationary epoch. More recent measurements by the WMAP and PLANCK missions have greatly improved
upon the angular resolution.}
\label{fig:COBEsky}
\end{figure}

\subsection{The Epoch of Reionization}
The Epoch of Reionization refers to a time much later in the thermal history of the universe when the neutral gas went from almost completely neutral to very ionized. It is suspected that the
universe remained neutral after the time of Recombination ($z \sim 10-20$). At this point the formation of the first stars in the universe would have produced enough UV radiation to liberate the electrons from their $-13.6\,eV$ (these UV photons are $\sim$4-5 times more energetic than light at the purple end of the rainbow) potential well in hydrogen. 
\begin{figure}[h]
\centering
\includegraphics[width=\columnwidth]{Figures/21-centimeter-cosmology.jpg}
\caption{History of the universe, highlighting the Epoch of Reionization.
credit: \href{http://discovermagazine.com/2014/april/12-first-light}{Roen Kelly, Discover Magazine, April 2014}}
\label{fig:EoR}
\end{figure}

\subsection{Evidence for Dark Matter and Dark Energy}

\subsection{Summary}
\begin{itemize}
\item Nucleosynthesis describes the abundances of the light elements 
	that we observe in the universe. By solving for the equilibrium
	population of photons, nucleons, and atoms as the universe cools,
	we find that by mass, 75\% is $\prescript{1}{}H$ 
	and 25\% is $\prescript{4}{}He$.

\item The CBR comes to us after the plasma has cooled into atoms. 
	Before this time, the photons and electrons are in thermal 
	equilibrium. Afterwards, the universe becomes mostly transparent 
	to this thermal radiation and the resulting photon field is 
	cooled isentropically as the universe expands. The 2.7\,K 
	blackbody radiation that we observe today is this cosmological relic.
\end{itemize}

\begin{figure}[h]
\centering
\includegraphics[width=\columnwidth]{Figures/I02-09-nucleosynthesis1.jpg}
\caption{Pictorial description of Nucleosynthesis \url{http://universe-review.ca/F02-cosmicbg.htm}}
\label{fig:NucleoSynthGraphic}
\end{figure}

\clearpage
\subsection{Further Reading}
\begin{itemize}
\item \href{http://www.einstein-online.info/spotlights/BBN/?set_language=en}{Big Bang Nucleosynthesis by Achim Weiss, at Einstein Online}
\item \href{http://www.goodreads.com/book/show/150131.The_First_Three_Minutes}{The First Three Minutes}, by Steven Weinberg.
\item \href{http://www.cambridge.org/resources/0521846560/7706_Saha%20equation.pdf}{Excerpt on Saha Equation from Hale Bradt's book}
\item \href{http://udp.classroom.tv/class/14841/lecture-8-thermodynamics-in-an-expanding-universe-freeze-out-big-bang-nucleosynthesis}{Thermo in an Expanding Universe} online lecture from the Perimeter Institute
\item \href{http://en.wikipedia.org/wiki/Physical_cosmology}{Wikipedia: Physical Cosmology}
\item \href{http://www.astro.ucla.edu/~wright/CMB.html}{Ned Wright's CMB page}
\item \href{http://discovermagazine.com/2014/april/12-first-light}{Discover magazine article on Epoch of Reionization}
\end{itemize}

