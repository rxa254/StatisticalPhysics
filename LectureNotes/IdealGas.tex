\section{Ideal Gas: April 16}
As a first look at the Ideal Gas model, we start by assuming that we have $N$ identical point like particles in a box. These particles do not interact with each other and, by the virtue of being point-like, we can ignore the rotational and vibrational energies which real molecules have.\\

We can then procede to solve for the quantum wavefunctions and energies just as we do for the standard "particle-in-a-box" problem (cf. K\&K, Ch.\,1, pp.\,9-10).
\begin{equation}
\hat{H} \Braket{\Psi} = E \Braket{\Psi}
\end{equation}

\begin{figure}[h]
\centering
\includegraphics[width=0.7\columnwidth]{Figures/ParticleBoxPlot.pdf}
\caption{Example wavefunction for a particle in a square 2D box 
	with sides of length $L$}
\end{figure}

For this potential (the box), the Hamiltonian is just that for a free 
particle:

\begin{equation}
\hat{H} = -\frac{\hbar^2}{2 m} \nabla^2
\end{equation}

This type of wave equation has solutions of the form $A \sin{x} + B \cos{x}$.
By including the boundary conditions that $\Braket{\Psi} = 0$ when
($x, y, z$) = $0$ or $L$, we can eliminate the cosine terms and we are left
with:
\begin{equation}
\Psi(x,y,z) = \mathcal{C} \sin(n_x \frac{\pi}{L} x) \sin(n_y \frac{\pi}{L} y) \sin(n_z \frac{\pi}{L} z)
\end{equation}
where $\mathcal{C}$ is the normalization constant.