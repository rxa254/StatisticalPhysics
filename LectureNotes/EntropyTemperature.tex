\section{Temperature and Entropy: April 9}

\subsection{Entropy}
\label{s:Entropy}
The number of microstates is a huge number! Its much easier to work with the
logarithm of such large numbers. The logarithm is also convenient since adding
systems together requires just adding their logarithms, rather than multiplying
the number of microstates.

So we define \textit{\textbf{entropy}} as:
\begin{equation}
\sigma \equiv log(g)
\end{equation}
This is a dimensionless (unitless) quantity (its just the logarithm 
of a large number...). \\

As we saw in the previous lecture, the combined system moves from its initial configuration
(where system 1 and 2 have the energies $U_1$ and $U_2$, respectively) into the one which
is \emph{overwhelmingly} likely. Recall that our conclusion from the discussion of thermal
equilibrium is that for macroscopic systems (e.g. with $N \gtrsim 10^{15}$...actually the
transition from micro to macro is a fuzzy concept, but this is an OK estimate for now), the
chances of the energy being even 1\,ppm different from the most probable value are astronomically unlikely (really? Yes - see the Shakespeare, monkeys, and typewriters
problem in this week's problem set).

\begin{equation}
\bigg(\frac{\partial \sigma_1}{\partial U_1}\bigg)_{N_1} = 
\bigg(\frac{\partial \sigma_2}{\partial U_2}\bigg)_{N_2}
\label{eq:Teq}
\end{equation}

This tendency of the system to always move into a configuration which maximizes
the number of accessible micro-states ($g$) is a just another way of stating the
2$^{\rm nd}$ Law of Thermodynamics: The entropy of a closed system will always 
increase until the system reaches equilibrium.

\epigraph{Turning and turning in the widening gyre \\
The falcon cannot hear the falconer; \\
Things fall apart; the centre cannot hold; \\
Mere anarchy is loosed upon the world,...}{\textit{William B. Yeats, 1919}}

\subsection{Temperature}
\label{s:Temperature}

Intuitively, we know that putting things in contact brings them to the same temperature
(cf. the "Brain Freeze" available at the Red Door Cafe). So we want this to be
implied by Eq.~\ref{eq:Teq}.

From Eq.~\ref{eq:Teq}, we could pick $\tau, 1/\tau, or -\tau,...$, but we want temperature
to correspond to our human definitions of it. Zero temperature should be very low energy
and temperature should rise as we put more kinetic energy into the particles, so we define
it like so:
\begin{equation}
\frac{1}{\tau} \equiv \bigg(\frac{\partial \sigma}{\partial U}\bigg)_{N}
\label{eq:T}
\end{equation}

\subsection{The Laws of Thermodynamics}

\subsubsection{The Zeroth Law}

\subsubsection{The First Law}

\subsubsection{The Second Law}

\subsubsection{The Third Law}



