\section{Free Energy: April 16}
Early on we dealt with isolated systems. An isolated system has a 
fixed energy, $U$. Once we bring this system into contact with a
thermal reservoir, this is no longer true. So what will happen?
Well, the fundamental assumption of statistical mechanics is still 
true, so the system still seeks to be in the most probable state. \\

Let's use the tools of our newly found canonical ensemble approach.
Since the system is at a fixed temperature, we can rewrite the
partition function as a sum over energies instead a sum over microstates,
by including $g$, the multiplicity:
\begin{align}
Z(\tau) &\equiv \sum_s e^{-\epsilon_s/\tau} \\
        &= \sum_{\epsilon_s} g(\epsilon_s) e^{-\epsilon_s/\tau} \\
        &= \sum_{\epsilon_s} e^{\sigma(\epsilon_s)} e^{-\epsilon_s/\tau}
\end{align}
Then we can define a new quantity, the \emph{free energy}, as
$f \equiv \epsilon - \tau \sigma$. To find the most probable state, we
can maximize the probability (\cref{eq:PofS}), expressed in terms 
of energy (using the same swap as above):
\begin{equation}
P(\epsilon) = \frac{g(\epsilon) exp(-\epsilon/\tau)}{Z}
\label{eq:Pepsilon}
\end{equation}
So we find that the most probable state is the one for which the free energy is minimized (i.e. $P$ is maximized when the argument of the exponential is minimized).

By averaging over the ensemble of microstates, we can write 
$F = \Braket{f(\epsilon)}$ (just as we did for the energy, $U$). Using also the result (cf. K\&K p. 71) that $F = -\tau \log Z$, we can see that the probability can be written simply as:
\begin{equation}
P(\epsilon) = exp((F - \epsilon)/\tau)
\end{equation}

We can also calculate a few useful \textbf{differential relations} with the free energy. Starting with the differential of F (and including the thermodynamic identity):
\begin{align}
dF &= dU - \tau d\sigma - \sigma d\tau \\
   &= -\sigma d\tau - p dV
\end{align}
we find the following two relations for the partial derivatives of $F$:
\begin{align}
\bigg(\frac{\partial F}{\partial \tau}\bigg)_V = -\sigma \\
\bigg(\frac{\partial F}{\partial V}\bigg)_\tau = -p
\label{eq:dFdX}
\end{align}

