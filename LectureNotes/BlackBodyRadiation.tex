\section{Planck Black Body: April 21}

Let's consider now a system which is similar to the previous two (Ideal Gas and small System in a large Reservoir), but with a significant twist. Instead of a system where the constituents have a mass, we want to study what happens inside of a 
black\,\footnote{Here, by 'black', I mean that the walls of the box are a 'perfectly' absorbing conductor. This is a little counterintuitive; you know that very good conductors (e.g. aluminum, copper, silver) are very reflective and \emph{not} perfectly absorbing. Nevertheless, this turns out to be a valid approximation for what we are looking into and, in fact, works well with boxes of almost any generic properties.} box in thermal equilibrium with a Reservoir. Instead of particles of a gas, we will be working with \textit{photons}, the massless quanta of electromagnetic radiation. In the end we would like to end up with all the now usual concepts: total energy, entropy, pressure (yes, even light has a 
pressure~\footnote{cf. solar sails}), and the energy distribution (i.e. can 
we predict what color something will be?). \\

We'll start off with the assumption\,\footnote{which will be qualified later} that the energy of each photon of angular frequency $\omega$($=2 \pi \nu$) will just be proportional to the frequency:
\begin{equation}
\epsilon = \hbar \omega
\end{equation}
To an extremely good approximation, photons do not interact with each other (i.e. you cannot deflect a laser beam with another laser beam). So there can, in principle, be many photons with the same energy inside the box. We'll denote the occupancy of each mode as $s$, and so the energy per mode will be $\epsilon_s = s \hbar \omega$. The total energy will then just be an appropriately weighted sum over all the available modes.\\

At a temperature $\tau$, the Boltzmann factor for a \textit{single} mode 
is $e^{-s \hbar \omega / \tau}$. So the partition function for that mode is:
\begin{align}
Z &= \sum_{s=0}^{\infty} e^{-s \hbar \omega / \tau} \\
  &= \frac{1}{1 - e^{-\hbar \omega / \tau}}
\label{eq:PartPlanck}
\end{align}
where we have used the relation $\sum_{s=0}^{\infty} x^s = 1/(1-x)$, 
valid when $x < 1$. \\

At this point we can utilize all of the tools we have developed during the study of the canonical ensemble in Chapter 3. We know that the probability 
distribution can just be computed from the Boltzmann factor and the partition 
function (cf.\,\cref{eq:PofS}):
\begin{equation}
P(s) = e^{-s \hbar \omega / \tau} \bigg[ 1 - e^{-\hbar \omega / \tau} \bigg]
\end{equation}
and to find the average occupancy of each mode we just multiply by $s$ and sum over all states:
\begin{align}
\Braket{s} &= \sum_{s=0}^{\infty} s~P(s) \\
           &= \frac{1}{Z} \sum_{s=0}^{\infty} s~e^{-s \hbar \omega / \tau}
\end{align}
We can perform the sum here by transforming it into a more fmailiar form. To do this note that:
\begin{equation}
\sum s~e^{-s x} = -\frac{d}{dx} \sum e^{-s x}
\end{equation}
and the right hand side can be summed in the same way as \cref{eq:PartPlanck}, above.
So finally we arrive at the expression for the average occupancy:
\begin{equation}
\boxed{\Braket{s} = \frac{1}{e^{\hbar \omega / \tau} - 1}}
\end{equation}
which is the \emph{Planck distribution} for the occupancy of a single mode in thermal equilibrium with a heat bath of temperature $\tau$. The average energy in each mode is:
\begin{equation}
\boxed{\Braket{\epsilon} = \Braket{s} \hbar \omega= 
\frac{\hbar \omega}{e^{\hbar \omega / \tau} - 1}}
\end{equation}


