\section{Planck Black Body: April 21 - 23}

Let's consider now a system which is similar to the previous two (Ideal Gas and small System in a large Reservoir), but with a significant twist. Instead of a system where the constituents have a mass, we want to study what happens inside of a 
black\,\footnote{Here, by 'black', I mean that the walls of the box are a 'perfectly' absorbing conductor. This is a little counterintuitive; you know that very good conductors (e.g. aluminum, copper, silver) are very reflective and \emph{not} perfectly absorbing. Nevertheless, this turns out to be a valid approximation for what we are looking into and, in fact, works well with boxes of almost any generic properties.} box in thermal equilibrium with a Reservoir. Instead of particles of a gas, we will be working with \textit{photons}, the massless quanta of electromagnetic radiation. In the end we would like to end up with all the now usual concepts: total energy, entropy, pressure (yes, even light has a 
pressure~\footnote{cf. solar sails}), and the energy distribution (i.e. can 
we predict what color something will be?). \\

\begin{figure}[h]
\centering
\includegraphics[width=0.3\columnwidth]{Figures/245px-photon_waves.png}
\caption{EM waves in a black box with sides of length $L$}
\end{figure}


We'll start off with the assumption\,\footnote{which will be qualified later} that the energy of each photon of angular frequency $\omega$($=2 \pi \nu$) will just be proportional to the frequency:
\begin{equation}
\epsilon = \hbar \omega
\end{equation}
To an extremely good approximation, photons do not interact with each other (i.e. you cannot deflect a laser beam with another laser beam). So there can, in principle, be many photons with the same energy inside the box. We'll denote the occupancy of each mode as $s$, and so the energy per mode will be $\epsilon_s = s \hbar \omega$. The total energy will then just be an appropriately weighted sum over all the available modes.\\

At a temperature $\tau$, the Boltzmann factor for a \textit{single} mode 
is $e^{-s \hbar \omega / \tau}$. So the partition function for that mode is:
\begin{align}
Z &= \sum_{s=0}^{\infty} e^{-s \hbar \omega / \tau} \\
  &= \frac{1}{1 - e^{-\hbar \omega / \tau}}
\label{eq:PartPlanck}
\end{align}
where we have used the relation $\sum_{s=0}^{\infty} x^s = 1/(1-x)$, 
valid when $x < 1$. \\

\subsection{The Planck Distribution}
At this point we can utilize all of the tools we have developed during the study of the canonical ensemble in Chapter 3. We know that the probability 
distribution can just be computed from the Boltzmann factor and the partition 
function (cf.\,\cref{eq:PofS}):
\begin{equation}
P(s) = e^{-s \hbar \omega / \tau} \bigg[ 1 - e^{-\hbar \omega / \tau} \bigg]
\end{equation}
and to find the average occupancy of each mode we just multiply by $s$ and sum over all states:
\begin{align}
\Braket{s} &= \sum_{s=0}^{\infty} s~P(s) \\
           &= \frac{1}{Z} \sum_{s=0}^{\infty} s~e^{-s \hbar \omega / \tau}
\end{align}
We can perform the sum here by transforming it into a more familiar form. To do this note that:
\begin{equation}
\sum s~e^{-s x} = -\frac{d}{dx} \sum e^{-s x}
\end{equation}
and the right hand side can be summed in the same way as \cref{eq:PartPlanck}, above.
So finally we arrive at the expression for the average occupancy:
\begin{equation}
\boxed{\Braket{s} = \frac{1}{e^{\hbar \omega / \tau} - 1}}
\label{eq:PlanckOccupy}
\end{equation}
which is the \emph{Planck distribution} for the occupancy of a single mode in thermal equilibrium with a heat bath of temperature $\tau$. The average energy in each mode is:
\begin{equation}
\boxed{\Braket{\epsilon} = \Braket{s} \hbar \omega= 
\frac{\hbar \omega}{e^{\hbar \omega / \tau} - 1}}
\end{equation}
Its interesting to look at the high and low temperature (or equivalently, the low and high frequency) limits of this expression. \\

At high temperatures, where 
$\tau \gg \hbar \omega$, we can expand the denominator ($e^x \simeq 1 + x$) and see
that $\Braket{\epsilon} \to \tau$. This is the 'classical limit'; the energy is just
proportional to the temperature and there is no evidence of quantization.\\

At low temperatures, where $\tau \ll \hbar \omega$, the denominator becomes large
and $\Braket{\epsilon} \to 0$. So a black box will tend towards zero energy in the
electromangetic field~\footnote{neglecting the ground state energy of 
$\frac{1}{2}\hbar \omega$ per mode} as well as zero occupancy. \\

This highlights another interesting difference between
a box of ideal gas molecules and a box of radiation: \emph{the photon number is
not a conserved quantity}. As we will see soon, the same is true for 
\textit{phonons}, the acoustic excitations in a solid.


\subsection{The Stefan-Boltzmann Law}
We would like to now move on to the main goal, which is to find the total energy,
entropy, etc. for the blackbody, summing over all the modes.

\begin{equation}
U = 2 \sum_n \Braket{\epsilon_n} = 
\sum_n \frac{\hbar \omega_n}{exp(\hbar \omega_n / \tau) - 1}
\end{equation}
where we have followed the convention from K \& K pp. 92-93 which describes
the accounting for the modes allowed in the box (similar reasoning as we use
for waves on a string and particle in a box). The factor of 2 comes from accounting
for the 2 polarizations of the radiation field and $\omega_n = n \pi c/L$.\\

We can replace the sum over the indices ($n_x, n_y, n_z$) with a triple integral
to make the computation easier. This is valid as long as the box is not small
with respect to the wavelength of the relavent radiation field ($L \gg c/\omega$).\\

To make the integral easier, we'll convert it from Cartesian coordinates to
spherical coordinates and replace the volume element $dn_x dn_y dn_z$, with
the spherical volume element $4 \pi\,n^2 dn$.
\begin{align}
U &= 2 \times \frac{1}{8} \times 4 \pi 
\int_{0}^{\infty} n^2 \frac{\hbar \omega_n}{exp(\hbar \omega_n / \tau) - 1} dn \\
  &= \frac{L^3 \tau^4}{\pi^2 \hbar^3 c^3} \int_{0}^{\infty} \frac{x^3}{e^x - 1} dx \\
  &= \frac{\pi^2 L^3 \tau^4}{15 \hbar^3 c^3}
\label{eq:BBenergy}
\end{align}
where the factor of 2 is for both polarizations, the $1/8$ because we only want to consider one octant of the spherical volume (where the wavenumbers are positive), and the $4 \pi$ covers the angular part of the volume integral. We then have made the substitution $x = \pi \hbar c n / L \tau$ to get the dimensionless 
integral. We can look up~\footnote{you can do it for yourself in a few steps if you want: multiply top and bottom by $e^{-x}$ and then recall our infinite sum for 
$1/(1-x)$} the integral in a book or Mathematica to find that its $\pi^4/15$.
The \emph{Stefan-Boltzmann constant} (the above prefactor) is defined as:
\begin{equation}
\sigma_B = \frac{\pi^2}{15}\frac{k_B^4}{\hbar^3 c^3}
\end{equation}


Its useful to also look at the spectral distribution of the energy. To get that
we can rewrite the integrand above as:
\begin{align}
\frac{U}{V} &= \int_{0}^{\infty} u_{\omega} d\omega \\
u_{\omega}  &= \frac{\hbar}{\pi^2 c^3} \frac{\omega^3}{e^{\hbar \omega/\tau} - 1}
\label{eq:PlanckSpectralDensity}
\end{align}
where we've moved the volume ($V = L^3$) over to the left hand side and used
our expression from above to replace $n$ with $\omega$. This
\textit{spectral density}, $u_{\omega}$, is called the
\emph{Planck Black Body Spectrum} and the theoretical and experimental work which
gave rise to it was the first evidence for $\hbar$ and was the birth of
Quantum Mechanics.

\begin{figure}[ht]
\centering
\includegraphics[width=0.7\columnwidth]{Figures/BlackbodySpectrum_loglog_150dpi_en.png}
\caption{Blackbody Radiation Intensity Spectral density. \\
	From \url{https://commons.wikimedia.org/wiki/File:BlackbodySpectrum_loglog_150dpi_en.png}}
\end{figure}

