\section{Boltzmann Factor: April 14}


\subsection{The Boltzmann Factor}
An important problem in Statistical Physics, is to find the probability
of finding a system in a state $s$ of energy $\epsilon_s$. We already
know how to do this for closed systems, but most systems are, in reality, open systems -- that is, they are in contact with a large thermal bath or reservoir. This might be the rest of the room (in the case of a tiny experiment) or the rest of the city (in the case that the system is a building) or even the rest of the universe (in astrophysical or cosmological cases). \\

So lets take our little system $\mathcal{S}$ and put in contact with our large reservoir $\mathcal{R}$ with initial energy $U_0$ and temperature $\tau_R$. Now we put our system in thermal (and later, mechanical) contact with the reservoir such that the new energy of the reservoir is $U_0 - \epsilon_s$. Now 
$\mathcal{R}$ can be in any of its $g_R(U_0 - \epsilon_s)$ microstates, just as we saw in last week's \textit{microcanonical} picture. \\

For a given state $s$, there is no degeneracy; the multiplicity, $g_s = 1$. So the probability of finding the total system ($\mathcal{S} + \mathcal{R}$) in the state where $\mathcal{S}$ has energy $\epsilon_s$, is just dependent on the multiplicity $g_R$: $P(s) \propto g_R(U_0 - \epsilon_s)$. Since 
$U_0 \gg \epsilon_s$, we can use the Taylor expansion to find a simplified form:

\begin{equation}
\log g_R = \sigma_R(U_0 - \epsilon_s) \simeq 
	\sigma_R(U_0) - 
	\epsilon_s \bigg(\frac{\partial \sigma_R}{\partial U_R}\bigg) +
	\mathcal{O}(\epsilon_s^2)
\end{equation}

Since $(\partial \sigma_R/\partial U_R) = 1/\tau_R$, we can rewrite the proportionality for the probability as:

\begin{equation}
P(s) \propto exp(-\epsilon_s/\tau_R)
\end{equation}

This is \textbf{one the most practically useful results in statistical physics}: it allows us to compute the relative probability of the system being in different energy states. This expression, $exp(-\epsilon_s/\tau_R)$, is called the
\emph{Boltzmann Factor}.


\subsection{Partition Function}
We would like to normalize the probability so that 
$\sum_s P(s) = 1$. To do this, we just sum over all possible energy states.

\begin{equation}
P(s) = \frac{exp(-\epsilon_s/\tau_R)}{\sum_s exp(-\epsilon_s/\tau_R} = 1
\end{equation}
and where we will define the \emph{Partition Function}\footnote{an oddly named function for sure} as
\begin{equation}
Z(\tau_R) \equiv \sum_s exp(-\epsilon_s/\tau_R
\end{equation}

\subsection{Pressure}

\subsection{Free Energy}


\textbf{Summary}

\begin{itemize}
\item The probability of finding a system (in microstate $s$ with
	energy $\epsilon_s$, in thermal equilibrium with a reservoir 
	at temperature $\tau_R$) is proportional to the 
	\emph{Boltzmann factor} $P(s) \propto exp(-\epsilon_s/\tau_R)$.

\item To normalize $P(s)$ properly ($\sum_{s} P(s) = 1$), we divide
	the Boltzmann factor by the sum over all energy states.
	$P(s) = exp(-\epsilon_s/\tau_R)/Z$, where 
	$Z(\tau_R) \equiv \sum_{s} exp(-\epsilon_s/\tau_R)$. $Z$ is called the
	partition function.

\item For a compressible system, if we change the volume slowly and by
	a small amount, the entropy won't change (\textit{isentropic}). This
	is defined as \textit{pressure}: $p = -(\partial U/\partial V)_\sigma$

\item The picture of closed system with an accessible number of microstates
	and associated entropy is the \emph{microcanonical ensemble}. In this picture, the energy, volume, and number of the system is fixed.

\item The \emph{canonical ensemble} is the similar picture, but with 
	the system now in thermal equilibrium with a heat bath. Consequently,
	the temperature is now fixed, but the energy of the system is not.

\end{itemize}




