\documentclass[11pt]{article}
\setlength{\oddsidemargin}{0in}
\setlength{\topmargin}{0.0in}
\setlength{\textwidth}{6.7in}
\setlength{\textheight}{8.5in}
%
%\input mydefs.tex
\def\vev#1{\left\langle #1\right\rangle}
\def\hb{\hfill\break}
%
\begin{document}
%
\centerline{\large\bf Physics 2c \hfill Assignment V \hfill  5/7/15}

\medskip
\begin{list}{}{\leftmargin 2.4cm \labelsep .2cm \labelwidth 2.2cm}
\item[{\bf Reading:}  \hfill ] This week: K\&K, Chapter 5 - Chemical Potential. \hb
                               and        K\&K, Chapter 6 - Ideal Quantum Gas. \hb
                               Next week: K\&K, Chapter 7 - Fermi \& Bose Gases.
\item[{\bf Problems:} \hfill ] Due in ph2 IN box, 4\,PM  Thursday, 5/14/15.
\end{list}

\hrule

\begin{description}
\item[{\bf V.1} ] {\bf Gas in a gravitational field.} K\&K 5.2, 5.3.

\item[{\bf V.2} ] {\bf Gibbs sum.} K\&K 5.6.

\item[{\bf V.3} ] {\bf Adsorption.} K\&K 5.8.\\
\\
\hrule

\item[{\bf V.4}] {\bf Relativistic Gas.} K\&K 6.4.

\item[{\bf V.5}] {\bf Gas with internal degrees of freedom.} K\&K 6.9.


\end{description}

\end{document}

\item[{\bf V.5} ] {\bf Relation of pressure and energy density; pressure dispersion.}
\begin{itemize}
\item[a)]  K\&K 6.7.
\item[b)] 
Using the results of the above problem, 
compute $\sigma_p^2 = \vev{(\Delta p)^2}$,
the dispersion in the pressure, 
and show that it is directly related to 
$\sigma_E^2 = \vev{(\Delta E)^2}$,
the dispersion in the energy.
\item[c)] 
Show that $\sigma_p^2 = \frac{2}{3V}\tau^2 \frac{\partial p}{\partial\tau}$.
\end{itemize}




\item[{\bf V.3} ] {\bf Convective isentropic equilibrium 
   of the atmosphere.} K\&K 6.11.

\item[{\bf V.4} ] {\bf Thermodynamic potentials and work.}
A substance has the following properties:
\begin{itemize}
\item[ ] \vskip -.2cm
\begin{itemize}
\item[{\it i})] 
At a particular constant temperature $\tau_0$,
the mechanical work it does in expanding from $V_0$ to $V$ is
$$W = \int^{V}_{V_0} p\,dV = N\tau_0 \log\left(\frac{V}{V_0}\right) . $$
\item[{\it ii})] 
The entropy at any temperature $\tau$ and volume $V$ is 
$\sigma = N \frac{V_0}{V}\left(\frac{\tau}{\tau_0}\right)^\alpha$,
where $\tau_0$ and $V_0$ are the constants of {\it i})
and $\alpha$ is another constant.
\end{itemize}
\item[{\bf a)}]
Calculate the Helmholtz free energy $F(V,\tau)$ relative to its value
at $F(V_0,\tau_0)$; that is, find $dF = F(V,\tau) - F(V_0,\tau_0)$.
[ Hint: The difference in $F$ between any two points
in the $(V,\tau)$ plane is the line integral of the partial derivatives
of $F$ along any convenient path connecting the two points.
As a proper thermodynamic potential, the difference is independent
of the path taken.]
\item[{\bf b)}]
What is the equation of state?
\item[{\bf c)}]
Find $U(V,\tau)$.
\item[{\bf d)}]
Find the work done in an isothermal expansion from $V_0$ to $V$
at an arbitrary temperature $\tau$.
\end{itemize}

\item[{\bf V.5} ] {\bf The R\"uchardt Experiment.} 
This simple experiment allows one to measure the ratio
$\gamma = C_p/C_V$ for an ideal gas.
A bottle of volume $V$ is filled with an ideal gas.
It has a precision-bore neck of area $A$,
into which is fitted a piston of mass $M$,
free to move up and down without friction
inside the neck; the gas cannot leak out of the bottle,
neck and piston enclosure.
The piston, once displaced from its equilibrium position
and released, will bounce up and down
at (angular) frequency $\omega$ 
with a small amplitude
(such that the total volume of the enclosure 
is only slightly varying); its displacement 
at any given time is given by $x$.
The enclosure is insulated, so
this process may be considered practically adiabatic. \\

\rightline{Over $\rightarrow$}

\unitlength=1in
\begin{center}
\begin{picture}(3,2)(0,0)
\put(1.4,1.8){\line(0,-1){0.8}}
\put(1.4,1.0){\line(-1,0){0.4}}
\put(1.0,1.0){\line(0,-1){0.8}}
\put(1.0,0.2){\line(1,0){1.0}}
\put(2.0,0.2){\line(0,1){0.8}}
\put(2.0,1.0){\line(-1,0){0.4}}
\put(1.6,1.0){\line(0,1){0.8}}
\put(1.4,1.4){\line(1,0){0.2}}
\put(1.4,1.5){\line(1,0){0.2}}
\put(1.7,1.3){\vector(0,1){0.3}}
\put(1.8,1.5){\makebox(0,0){x}}
\put(1.5,0.5){\makebox(0,0){V}}
\put(1.3,1.45){\makebox(0,0){M}}
\put(1.5,1.8){\makebox(0,0){A}}
\end{picture}
\end{center}
\begin{itemize}
\item[a)] What is the pressure $p$ of the gas
inside the enclosure?
\item[b)] Derive a differential equation
for $x(t)$ in terms of $\gamma$, $p$, $M$, $V$, $A$.
\item[c)] From the solution to part (b),
derive an expression for $\gamma$ 
in terms of $p$, $M$, $V$, $A$, and the period
of oscillation, $T$.
\end{itemize}

\item[{\bf V.6} ] {\bf Thermodynamic Potentials.}
The handout on Thermodynamic Potentials contains
an awful lot of formulas. Some are bound to be wrong. In fact,
I have {\it deliberately} introduced three errors in the formulas.
Find them! 
(You need not hand in the solution; it won't be graded.
This problem is worth 0 points; your reward is an error-free 
collection of formulas).

\item[{\bf V.2} ] {\bf 2D Adsorbed gas.}
A dilute ideal gas of $N$ atoms is at temperature $\tau$
in a container of volume $V$.
$N^\prime\ll N$ atoms are adsorbed on one of the walls,
whose area is $A$.
These adsorbed atoms are confined to the wall in the normal direction
with a binding energy $\epsilon_0$, but are otherwise
free to move in the plane of the wall
(\ie, they form a two-dimensional ideal gas).
\begin{itemize}
\item[a)] 
If $N^\prime$ is fixed, use the canonical ensemble 
to find the chemical potential of the adsorbed atoms.
\item[b)] 
The gas on the wall is allowed to reach equilibrium
with the gas volume.
Find the average surface density of adsorbed atoms,
$\vev{N^\prime}/A$, as a function of $\tau$ and $p$,
the pressure of the contained gas.
\unitlength=1cm
\begin{center}
\begin{picture}(10,3)(0,0)
\put(3.5,1){\framebox(3,2){}}
\put(5,2.5){\makebox(0,0){(gas)}}
\put(5,2.0){\makebox(0,0){$N,\tau,V$}}
\put(3.9,1.1){\circle*{0.2}}
\put(4.4,1.1){\circle*{0.2}}
\put(5.3,1.1){\circle*{0.2}}
\put(5.9,1.1){\circle*{0.2}}
\put(6.3,1.1){\circle*{0.2}}
\put(1.3,1.1){\makebox(0,0){(adsorbed atoms)}}
\put( 3, 1.1){\vector(1,0){0.75}}
\put(1.3,0.7){\makebox(0,0){$N^\prime,\tau,A$}}
%\put(8,0.5){\makebox(0,0){$N^\prime,\tau,A$}}
%\put(7,0.5){\vector(-3,1){1.5}}
\end{picture}
\end{center}
\end{itemize}


\item[\underline{\bf Problem 3.}] {\bf Fermionic Photon Gas}.
Consider a large box of volume $V$ whose walls are held at 
constant temperature $\tau$.
The box contains photons that are in thermal equilibrium with the walls.
Suppose that, contrary to reality, the photons obey the Pauli Exclusion 
Principle, so that each mode can only have zero or one photon in it.
\begin{itemize}
\item[{[3]}\ a)] 
What is the thermal average of the number of photons 
in a single mode of frequency $\omega$?
\item[{[4]}\ b)] 
What is the energy spectrum of the light inside the box?
Compare this result with the one for ordinary black-body radiation.
Give a simple physical explanation for any similarities or differences.
\item[{[3]}\ c)] 
What are the total number of photons and total energy
stored inside the box?
How do you expect these quantities to compare with 
those for ordinary black-body radiation?
(Leave any integrals you can't do in dimensionless form).
\end{itemize}


