\documentclass[11pt]{article}
\setlength{\oddsidemargin}{0in}
\setlength{\topmargin}{0.0in}
\setlength{\textwidth}{6.7in}
\setlength{\textheight}{8.5in}
%
%\input mydefs.tex
\def\vev#1{\left\langle #1\right\rangle}
\def\hb{\hfill\break}
%
\begin{document}
%
\centerline{\large\bf Physics 2c \hfill Assignment VII \hfill  5/22/15}

\medskip
\begin{list}{}{\leftmargin 2.4cm \labelsep .2cm \labelwidth 2.2cm}
\item[{\bf Reading:}  \hfill ] This week: K\&K, Chapter 8 - Heat and Work \\
                               Next week: Lecture Notes: Early Universe
\item[{\bf Problems:} \hfill ] Due in Ph2 IN box in East Bridge, Thu 5/28/15, 4\,pm.
\end{list}

\begin{description}

\item[{\bf VII.1} ] {\bf Gasoline Engine.} 
The operation of a gasoline engine can be modeled 
as an ideal gas undergoing the following cycle of 
reversible transformations:\\
\unitlength=1cm
\begin{center}
\begin{picture}(12,2.4)(0,0)
\put(1.0,0.0){\line(1,0){1.8}}
\put(1.0,1.2){\line(1,0){1.8}}
\put(1.0,0.0){\line(0,1){1.2}}
\put(2.4,0.0){\line(0,1){1.2}}
\put(2.5,0.0){\line(0,1){1.2}}
\put(1.9,2.1){\makebox(0,0){(i)}}
\put(1.8,1.6){\vector(0,-1){0.4}}
\put(1.9,1.6){\vector(0,-1){0.4}}
\put(1.6,0.5){$V_1$}
\put(1.2,1.4){$Q_a$}
%
\put(3.8,0.0){\line(1,0){1.8}}
\put(3.8,1.2){\line(1,0){1.8}}
\put(3.8,0.0){\line(0,1){1.2}}
\put(5.2,0.0){\line(0,1){1.2}}
\put(5.3,0.0){\line(0,1){1.2}}
\put(4.7,2.1){\makebox(0,0){(ii)}}
\put(5.3,0.6){\vector(1,0){0.4}}
%
\put(6.6,0.0){\line(1,0){1.8}}
\put(6.6,1.2){\line(1,0){1.8}}
\put(6.6,0.0){\line(0,1){1.2}}
\put(8.2,0.0){\line(0,1){1.2}}
\put(8.3,0.0){\line(0,1){1.2}}
\put(7.5,2.1){\makebox(0,0){(iii)}}
\put(7.4,1.2){\vector(0,1){0.4}} 
\put(7.5,1.2){\vector(0,1){0.4}}
\put(6.8,1.4){$Q_c$}
\put(7.2,0.5){$V_2$}
%
\put(9.4,0.0){\line(1,0){1.8}}
\put(9.4,1.2){\line(1,0){1.8}}
\put(9.4,0.0){\line(0,1){1.2}}
\put(11.0,0.0){\line(0,1){1.2}}
\put(11.1,0.0){\line(0,1){1.2}}
\put(10.3,2.1){\makebox(0,0){(iv)}}
\put(11.0,0.6){\vector(-1,0){0.4}}
\end{picture}
\end{center}

\begin{itemize}
\item[(i)] 
Heating at constant volume:  
the gas goes from
$({\tau} = {\tau_{A}},~{V} = {V}_1)$ 
to $({\tau_{B}}, {V}_{1})$.
\item[(ii)] 
Isentropic expansion (entropy constant): 
the gas goes from $({\tau_{B}}, {V}_{1})$ 
to $({\tau_{C}}, {V}_{2})$.
\item[(iii)] 
Cooling at constant volume:
the gas goes from $({\tau_{C}}, {V}_{2})$
to $({\tau_{D}}, {V}_{2})$.
\item[(iv)] 
Isentropic compression:
the gas goes from $({\tau_{D}}, {V}_{2})$ 
back to $({\tau_{A}}, {V}_{1})$.
\end{itemize}

\begin{itemize}
\item[a)] 
Calculate the engine's efficiency; i.e.,
the ratio ${\eta} = {W}/{Q}_{a}$  
of the work done by the gas in one cycle, 
{\it W}, to the heat absorbed 
during step (i), ${Q}_{a}$. 
[Hint:  the internal energy, ${U}$,
is the same at the beginning of step (i) 
and at the end of step (iv).]
\item[b)] 
Use the result of part a) to obtain an upper bound for
the efficiency of a real engine operating with
compression ratio $V_2/V_1 = 10$.
\end{itemize}

\item[{\bf VII.2} ] {\bf Photon Carnot Engine.} K\&K 8.3.

\item[{\bf VII.3} ] {\bf Thermal Pollution} K\&K 8.5.

\item[{\bf VII.4} ] {\bf Air Conditioning} K\&K 8.6.

\item[{\bf VII.5} ] {\bf Geothermal Energy} K\&K 8.7.


\end{description}

\end{document}



\item[{\bf VII.4} ] {\bf Heat engine -- refrigerator cascade.} K\&K 8.4.

\item[{\bf VII.5} ] {\bf Heat extraction.} K\&K 8.8 and 8.9.

\item[{\bf VII.5} ] {\bf Absorption refrigerator.} K\&K 8.2.
%\item[{\bf VII.4} ] {\bf Thermal ionization of hydrogen.} K\&K 9.2.

%\item[{\bf VII.5} ] {\bf Particle-antiparticle equilibrium.} K\&K 9.5.


