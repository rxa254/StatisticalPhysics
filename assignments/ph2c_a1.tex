\documentclass[11pt]{article}
\setlength{\oddsidemargin}{0in}
\setlength{\topmargin}{0.0in}
\setlength{\textwidth}{6.7in}
\setlength{\textheight}{8.5in}
%
%\input mydefs.tex
\def\vev#1{\left\langle #1\right\rangle}
%
\begin{document}
%
\centerline{\large\bf Physics 2c \hfill Assignment I \hfill  4/2/15}

\medskip
\begin{description}
\item[{\bf Reading:} ] Kittel and Kroemer, Intro, Chapter 1, Appendix A \\
                       For next week: Chapter 2 and Appendix B.
\item[{\bf Problems:} \hfill ] Due in Ph2 IN-box in Bridge Annex, 5PM  Thursday, 4/9/15.
\end{description}


\medskip

\begin{description}

\item[{\bf I.1} ]
{\bf Bad Robot} \\
A mis-programmed robot has a probability $0<p<1$ to step forward ($+x$)
by an amount $\delta$ and a probability $q=1-p<p$ to step backwards ($-x$)
by the same amount.
\begin{itemize}
\item[\bf a)] If $x$ is the robot's distance
from its starting point, what is the probability to find it
at $x=n\delta$ after $N$ steps?
What is $\vev{x}$?
$\vev{x^2}-\vev{x}^2$?
\item[\bf b)] What is the limiting form of a)
for $N\rightarrow\infty, \, |n|\ll N$?
\end{itemize}

\item[{\bf I.2} ]
{\bf Gas in a container} \\
Consider a box of volume $V_T$ containing $N_T$ atoms,
each of which is equally likely to be anywhere in the box.
Let $V\le V_T$ be a sub-volume of the box.
\begin{itemize}
\item[\bf a)] What is the average number of atoms in $V$?
\item[\bf b)] What is the standard deviation in the number of atoms in $V$?
\item[\bf c)] Sketch the probability distribution
              for the number of atoms in $V$
              for $V=V_T$, $V=\frac{1}{2}V_T$, and $V = 10^{-6}V_T$.
              Assume $N_T = 10^{23}$.
\end{itemize}


\item[{\bf I.3} ] 
{\bf Alpha particle decay} \\
Kittel and Kroemer C.1.

\item[{\bf I.4} ] 
{\bf Classical Harmonic Oscillator Probability Distribution} \\
The displacement $x$ of a classical simple harmonic oscillator as a function of time
is given by $x = A\cos(\omega t+\phi)$
where $\omega$ is the angular frequency, $A$ the amplitude of oscillation,
and $\phi$ is an arbitrary initial parameter in the range $0\le \phi\le 2\pi$.\\
Find the probability $P(x)dx$ that the displacement of the 
oscillator, at a random instant, is fround to lie 
between $x$ and $x + dx$.
% The solution is in Liboff p 201.


% \item[{\bf I.5} ] 
% {\bf Approach to Gaussian distribution.}
% Kittel and Kroemer C.2 (p 458).

\end{description}

\end{document}

\item[{\bf I.3} ]
{\bf The Poisson Distribution.}
\begin{itemize}
\item[\bf a)] The probability $W(n)$ that an event characterized by a probability $p$
              occurs $n$ times in $N$ trials is given by the 
              binomial distribution $W(n) = C^N_n p^n (1-p)^{N-n}$.
              Consider the limit where $N\to \infty$, $p \ll 1$, with $\bar n = Np$ finite.
              Show that in these limits, $W(n)$ is approximated by
              the Poisson distribution $P_{\bar n}(n) = {\bar n}^n e^{-\bar n}/n!$.
\item[\bf b)] Show that the Poisson distribution is properly normalized,
              in the sense that $\sum_{n=0}^{\infty} P_{\bar n}(n) = 1$.
\item[\bf c)] Calculate $\vev{n}$ and $\sigma = \sqrt{\vev{n^2}-\vev{n}^2}$.
              Trick: consider $d P_{\bar n}(n) / d\bar n$.
\end{itemize}

\item[{\bf I.3} ]
{\bf Choices in Love.}
A young man, who lives at location $A$ of the city street plan
shown in the figure, walks daily to the home of his fianc\'ee,
who lives $m$ blocks east and $m$ blocks north of $A$, at location $B$.
Because he is always anxious to see his fianc\'ee, his route always
approaches $B$, ie, he never doubles back.
In how many different ways can he go from $A$ to $B$?

\unitlength=2pt
\begin{center}
\begin{picture}(50,50)(-12,-12)
\thinlines
\put( 0, 0){\line(1,0){36}}
\put( 0, 8){\line(1,0){36}}
\put( 0,16){\line(1,0){36}}
\put( 0,24){\line(1,0){36}}
\put( 0,32){\line(1,0){36}}
\put( 0, 0){\line(0,1){32}}
\put(12, 0){\line(0,1){32}}
\put(24, 0){\line(0,1){32}}
\put(36, 0){\line(0,1){32}}
\thicklines
\put( 0, 0){\vector(1,0){6}}
\put( 6, 0){\line(1,0){6}}
\put(12, 0){\vector(0,1){12}}
\put(12,12){\line(0,1){12}}
\put(12,24){\vector(1,0){12}}
\put(24,24){\line(1,0){12}}
\put(36,24){\vector(0,1){ 4}}
\put(36,28){\line(0,1){ 4}}
\put( -6, 0){$A$}
\put( 40,30){$B$}
\put(-16,16){$n=4$}
\put( 15,-6){$m=3$}
\end{picture}
\end{center}


