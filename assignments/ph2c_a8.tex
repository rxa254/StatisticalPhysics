\documentclass[11pt]{article}
\setlength{\oddsidemargin}{0in}
\setlength{\topmargin}{0.0in}
\setlength{\textwidth}{6.7in}
\setlength{\textheight}{8.5in}
%
%\input mydefs.tex
\def\vev#1{\left\langle #1\right\rangle}
\def\hb{\hfill\break}
%
\begin{document}
%
\centerline{\large\bf Physics 2c \hfill Assignment VIII \hfill  5/28/15}

\medskip
\begin{list}{}{\leftmargin 2.4cm \labelsep .2cm \labelwidth 2.2cm}
\item[{\bf Reading:}  \hfill ] This week: Lecture Notes: Early Universe \\
                               Next week: 
\item[{\bf Problems:} \hfill ] Due in Ph2 IN box in East Bridge, Thu 6/4/15, 4\,pm.
\end{list}

\begin{description}

\item[{\bf VIII.1} ] {\bf Thermal Ionization of Hydrogen: K\&K 9.2} \\
\textbf{(part (a) \emph{only})}\\
This derivation of the Saha equation can be used to understand the
recombination process at $z \sim 1100$ which preceded the decoupling
of photons from matter.

\item[{\bf VIII.2} ]

\item[{\bf VIII.3} ]

\item[{\bf VIII.4} ]

\item[{\bf VIII.5} ]

\end{description}

\end{document}




%\item[{\bf VII.4} ] {\bf Thermal ionization of hydrogen.} K\&K 9.2.

%\item[{\bf VII.5} ] {\bf Particle-antiparticle equilibrium.} K\&K 9.5.


