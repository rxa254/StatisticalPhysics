\documentclass[11pt]{article}
\setlength{\oddsidemargin}{0in}
\setlength{\topmargin}{0.0in}
\setlength{\textwidth}{6.7in}
\setlength{\textheight}{8.5in}
%
%\input mydefs.tex
\def\vev#1{\left\langle #1\right\rangle}
\def\hb{\hfill\break}
%
\begin{document}
%
\centerline{\large\bf Physics 2c \hfill Assignment VIII \hfill  5/28/15}

\medskip
\begin{list}{}{\leftmargin 2.4cm \labelsep .2cm \labelwidth 2.2cm}
\item[{\bf Reading:}  \hfill ] This week: Lecture Notes: Early Universe + K\&K Ch.9 \\
                              Next week: lecture notes
\item[{\bf Problems:} \hfill ] Due in Ph2 IN box in East Bridge, Thu 6/4/15, 4\,pm.
\end{list}

\begin{description}

\item[{\bf VIII.1} ] {\bf Thermal Ionization of Hydrogen: K\&K 9.2} \\
\textbf{(part (a) \emph{only})}\\
This derivation of the Saha equation can be used to understand the
recombination process at $z \sim 1100$ which preceded the decoupling
of photons from matter.


\item[{\bf VIII.2} ] {\bf Big Bang Nucleosynthesis}\\
As discussed in the lectures, the abundances of the light elements
(deuterium, helium-3, helium-4, lithium) can be calculated from knowing
the thermal history of the universe in the 10\,s~$<$~t~$<$~1000\,s time
period. In order for these elements to form, it must be that an
appreciable amount of deuterium (1 proton + 1 neutron) can form
without immediately being thermally ionized.

At what temperature does the formation of deuterium become possible?



\item[{\bf VIII.3} ] {\bf The Time of Last Scattering}\\
After $t \sim 1000$\,s, the light elements have formed,
but there are still enough photons with $E > 13.6$\,eV to thermally
ionize hydrogen. At what temperature does the ratio of ionized hydrogen 
($H^{+}$) to neutral hydrogen drop below 1\% ?


\item[{\bf VIII.4} ] {\bf Thermal History of the Universe} \\
From Section 15.4 of the lecture notes, follow the argument for the
dynamical expansion of the radiation dominated universe. Show that

\begin{equation}
T(t) = \bigg(\frac{3 c^3}{32 \pi G \sigma_B}\bigg)^{1/4} \frac{1}{\sqrt{t}}
\end{equation}

where $\sigma_B$ is the Stefan-Boltzman constant, $c$ is the speed of light, and $G$ is Newton's Gravitational constant. Note that this formula gives an answer a factor of $\sim$2 too high from reality, due to the fact that we are neglecting the existence of other relativistic particles.

\end{description}

\end{document}




