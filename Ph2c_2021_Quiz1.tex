\documentclass{article}
\usepackage{geometry}        
\geometry{letterpaper}    
\usepackage[parfill]{parskip}  
\usepackage{graphicx}
\usepackage{amssymb}
\usepackage{amsthm}
\usepackage{epstopdf}
 \usepackage{amsmath,amssymb,amsfonts}
\usepackage{geometry}      
\geometry{letterpaper}  
\usepackage{mathtools}
 \usepackage{hyperref}
 \usepackage{cancel}
 \usepackage{epsfig}
% \usepackage{color}
 \usepackage[nosort]{cite}
 \usepackage[T1]{fontenc}
 \usepackage{lipsum}       % for sample text
 \usepackage{multicol}
 \usepackage{float}
\usepackage{changepage}
  \usepackage{soul}
\usepackage[usenames, dvipsnames]{color}
\usepackage{marginnote}
\usepackage{background}

\newtheorem{theo}{Theorem}[section]
\newtheorem*{cor}{Corollary}
\newtheorem*{prop}{Proposition}
\newtheorem{lemma}{Lemma}[theo]
\newtheorem*{defi}{Definition}
\theoremstyle{defi}
\newtheorem*{eg}{Example}


\newtheorem*{theo-no}{Theorem}
  \newcommand{\be}{\begin{equation}}
  \newcommand{\ee}{\end{equation}}
  \newcommand{\bea}{\begin{eqnarray}}
\newcommand{\eea}{\end{eqnarray}}
\newcommand{\e}{\epsilon}
\newcommand{\la}{\langle}
\newcommand{\ra}{\rangle}
\newcommand{\abs}[1]{\left|{#1}\right|}
\newcommand{\tr}[1]{\text{Tr}\left[{#1}\right]}
\newcommand{\norm}[1]{\left\|{#1}\right\|}
\newcommand{\bra}[1]{\langle{#1}|}
\newcommand{\ket}[1]{|{#1}\rangle}
\renewcommand{\L}{\mathcal{L}}
\newcommand{\R}{\mathbb{R}}
\newcommand{\C}{\mathbb{C}}
\newcommand{\I}{\mathcal{I}}
\renewcommand{\it}{\textit}
\newcommand{\bt}{\textbf}
\newcommand{\aikeq}{\textcolor{red}}
\newcommand{\aikec}{\textcolor{NavyBlue}}
\newcommand{\PI}{\textcolor{ForestGreen}}
\newcommand{\comment}[1]{\marginpar{\color{red}#1}}
\newcommand{\setle}{\subseteq}
\newcommand{\meas}[1]{\mu\left\{{#1}\right\}}
\newcommand{\ms}{(\Omega, \Sigma, \mu)}
\newcommand{\alge}[1]{\mathfrak{#1}}
 \begin{document}  
%%%%%%%%  
\SetBgContents{Spring 2021}
\SetBgScale{1.2}
\SetBgAngle{0}
\SetBgPosition{current page.north east}
\SetBgHshift{-2.5cm}
\SetBgVshift{-1cm}

\title{\vspace{-3cm}  \bt{Ph2c - Quiz 1}\\
\begin{large}
Due: April 19 at 4pm PDT
\end{large}
} 
\date{\vspace{-5ex}}
\maketitle

\begin{center}
    {\Large DO NOT PROCEED TO THE NEXT PAGE BEFORE YOU ARE READY TO TAKE THIS QUIZ.}
\end{center}

\vspace{15mm}

This is a \textbf{"LIMITED" OPEN-BOOK} quiz. Only the textbook (Kittel\&Kroemer), recitation notes, homeowork solutions (including your own) and lecture notes are allowed. You may use a calculator for simple, algebraic calculations. It must be worked out on your own, without collaboration with/from others. A late quiz will not be accepted except by prior arrangement with the Head TA and Instructor.

\vspace{5mm}
The quiz contains \textbf{3 PROBLEMS}, each worth 10 points. The time limit is \textbf{2.0 HOURS}, in one continuous sitting. A 15 minute break beyond the 2 hours is permitted during this period, as long as you are not working on the exam during this time. 

\vspace{5mm}
The quiz is \textbf{DUE} Monday Apr. 21 at 4PM and should be turned in via GradeScope. 

\vspace{5mm}
Please show as much work as you can for partial credits.

\pagebreak

\begin{enumerate}
    \item \textbf{Probability in Poker (10 pts)}  
    
    A deck of poker cards contains 52 cards in four different suits. Each suit consists of 13 cards with values ace, king, queen, jack, 10, 9, 8, ..., 3 and 2. At every deal, five cards are randomly drawn from the entire deck of 52 cards. That is, every deal is independent of each other.
    
        \begin{enumerate}
            \item (2 pts) Calculate the number of possible all five-card poker hands. Ignore the order of the cards in the hand.
            \item (2 pts) A royal flush consists of five highest-ranking cards (ace, king, queen, jack, 10) of any one of the four suits. What is the probability $p$ of being dealt a royal flush (on the first deal)?
            \item (3 pts) What is the probability of being dealt no royal flush after a total of 5 deals? You may express your answer in terms of $p$.
            \item (3 pts) What is the probability of being dealt royal flush at least twice, after a total of 5 deals? You may express your answer in terms of $p$.
        \end{enumerate}
    \vspace{15mm}
    
    \item \textbf{Shannon Entropy (10 pts)} 
    
        Suppose we receive messages of length $N$, each message consisting of a string of symbols $a$ and $b$, for example,
        \begin{equation}
            abbabababaabbbbbababababbaa...
        \end{equation}
        
        Suppose  $a$ occurs with probability $p$ and $b$ occurs with probability $1-p$. $0\leq p \leq1$.
        
        \begin{enumerate}
            \item (4pts) When $N$ is large, most messages will contain $pN$ $a$'s and $(1-p)N$ $b$'s. Show that the number of such messages are approximately 
            \begin{equation}
                2^{N S},
            \end{equation}
            where 
            \begin{equation}
                S = -p\log_2{p}-(1-p)\log_2{(1-p)}.
            \end{equation}
            $S$ is called the \textbf{Shannon entropy} per letter. It is roughly the number of bits of information per letter.
            
            \item (3pts) For what values of $p$ is the number of bits (the Shannon entropy $S$) per letter smallest? For what values of $p$ is it largest?
            
            \item (3pts) Now instead of $a$ and $b$, suppose we have an alphabet with $k$ letters $a_1,...,a_k$, and the probability to observe $a_i$ is $p_i$. Show that when $N$ is large, the number of possible messages is approximately $2^{N S}$, where
            \begin{equation}
                S = - \sum_{i=1}^k{p_i \log_2{p_i}}.
            \end{equation}
        \end{enumerate}
        
    \vspace{15mm}
    
    \item \textbf{Thermal Equilibrium of Spin Systems (10 pts)} 
        
        We start with two separate spin systems with particle numbers $N_{1}, N_{2}$ and initial spin excess $s_{1}^{i}, s_{2}^{i}$. The spins all have magnetic moment $\mu$ in magnetic field $B$, and the energy of a single spin is $U=\pm 2 \mu B$. The ratio of particle numbers are $N_{2} / N_{1}=2$ while the initial energies are related by $U_{2}^{i} / U_{1}^{i}=\frac{1}{2}$. For this problem, assume that the spin excess $s_1^i\ll N_1$ and $s_2^i \ll N_2$, so that the Gaussian approximation is appropriate for the multiplicity.

            (a) (2 pts) What is the initial ratio of entropies for the two systems? Express your answer in terms of $s_1^i, s_2^i, N_1$ and $N_2$. Hint: In the expression of the entropy $\sigma_1^i$, what are the leading order terms in $N_1$ and $s_1^i/N_1$?
                
            (b) (2 pts) What is the initial ratio of temperatures for the two systems? Express your answer as a numerical value.
                
            \vspace{5mm}
            The two systems are now brought into thermal contact with each other and allowed to exchange energy, but not particles. After they have reached equilibrium, answer the following questions.
                
            (c) (2 pts) What is the final ratio of energies for the two systems? Express your answer as a numerical value.
                
            (d) (1 pt) What is the final ratio of temperatures for the two systems? Express your answer as a numerical value.
                
            (e) (2 pts) What is the fractional change in the total entropy $\Delta \sigma / \sigma_{i}=\frac{\sigma_{f}-\sigma_{i}}{\sigma_{i}}$? Express your answer in terms of $s_1^i, s_2^i, N_1$ and $N_2$. Hint: It'd better be $\geq 0$.
                
            (f) (1 pt) Explicitly calculate the fractional change in entropy from part (e) for the following conditions: $N_{2}=2 \times 10^{22}, N_{1}=1 \times 10^{22}, s_{2}^{i}=1 \times 10^{20}, s_{1}^{i}=2 \times 10^{20}$.
\end{enumerate}



\end{document}
